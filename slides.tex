% -*- compile-command: "make slides.pdf" -*-
\documentclass{beamer}
\usepackage{tikz}
\usepackage[all]{xy}
\usepackage{amsmath,amssymb}
\usepackage{hyperref}
\usepackage{graphicx}
\usepackage{algorithmic}
\usepackage{multirow}

\DeclareMathOperator*{\argmin}{arg\,min}
\DeclareMathOperator*{\Lik}{Lik}
\DeclareMathOperator*{\PoissonLoss}{PoissonLoss}
\DeclareMathOperator*{\Peaks}{Peaks}
\DeclareMathOperator*{\Segments}{Segments}
\DeclareMathOperator*{\argmax}{arg\,max}
\DeclareMathOperator*{\maximize}{maximize}
\DeclareMathOperator*{\minimize}{minimize}
\newcommand{\sign}{\operatorname{sign}}
\newcommand{\RR}{\mathbb R}
\newcommand{\ZZ}{\mathbb Z}
\newcommand{\NN}{\mathbb N}

% Set transparency of non-highlighted sections in the table of
% contents slide.
\setbeamertemplate{section in toc shaded}[default][100]
\AtBeginSection[]
{
  \setbeamercolor{section in toc}{fg=red} 
  \setbeamercolor{section in toc shaded}{fg=black} 
  \begin{frame}
    \tableofcontents[currentsection]
  \end{frame}
}

\begin{document}

\title{Labeled Optimal Partitioning}

\author{
  Toby Dylan Hocking\\
  toby.hocking@nau.edu\\
  joint work with Anuraag Srivastava}

\date{6 June 2020}

\maketitle

\section{Demo}

% from https://tex.stackexchange.com/questions/160825/modifying-margins-for-one-slide
\newcommand\Wider[2][3cm]{%
\makebox[\linewidth][c]{%
  \begin{minipage}{\dimexpr\textwidth+#1\relax}
  \raggedright#2
  \end{minipage}%
  }%
}

\input{figure-candidates}

\end{document}
